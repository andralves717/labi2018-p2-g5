\documentclass{report}
\usepackage[T1]{fontenc} % Fontes T1
\usepackage[utf8]{inputenc} % Input UTF8
\usepackage[backend=biber, style=ieee]{biblatex} % para usar bibliografia
\usepackage{csquotes}
\usepackage[portuguese]{babel} %Usar língua portuguesa
\usepackage{blindtext} % Gerar texto automaticamente
\usepackage[printonlyused]{acronym}
\usepackage{hyperref} % para autoref
\usepackage{graphicx}
\usepackage{indentfirst}
\bibliography{bibliografia.bib}

\begin{document}

%%
% Definições
%
\def\titulo{Misturador de Músicas}
\def\data{15/06/2018}
\def\autores{André Alves, Daniel Correia, Pedro Almeida, Pedro Valente}
\def\autorescontactos{(88811) andr.alves@ua.pt, (88753) dcorreia@ua.pt,\\
 (89205) pedro22@ua.pt, (88858)pedro.valente@ua.pt}
\def\departamento{Departamento de Electrónica, Telecomunicações e Informática}
\def\empresa{Universidade De Aveiro}
\def\logotipo{img/ua.pdf}
%

%%%%%% CAPA %%%%%%
%
\begin{titlepage}

\begin{center}
%
\vspace*{50mm}
%
{\Huge \titulo}\\ 
%
\vspace{10mm}
%
{\Large \empresa}\\
%
\vspace{10mm}
%
{\LARGE \autores}\\ 
%
\vspace{30mm}
%
\begin{figure}[h]
\center
\includegraphics{\logotipo}
\end{figure}
%
\vspace{30mm}
\end{center}
%

\end{titlepage}

%%  Página de Título %%
\title{%
{\Huge\textbf{\titulo}}\\
{\Large \departamento\\ \empresa}
}
%
\author{%
    \autores \\
    \autorescontactos
}
%
\date{\data}
%
\maketitle

\pagenumbering{arabic}


\tableofcontents
\listoffigures    % descomentar se necessário



\chapter{Introdução}
\label{chap.introducao}
O objetivo do projeto é a criação de um sistema que permita criar músicas através da
composição de pedaços de música. O projeto funciona a partir de um servidor do XCOA.
O sistema é composto por vários componentes, componentes estes que são:
\begin{itemize}
	\item Interface Web
	\item Aplicação Web
	\item Persistência
	\item Gerador de Músicas
\end{itemize}	

A Interface Web (\autoref{chap.interface}) é composta por quatro páginas HTML cada uma com 
um propósito prório a explicar mais à frente.

A página Início é uma pequena página inicial na qual o utilizador poderá ver 
aquilo que pode fazer neste site, assim como algumas informações sobre os autores e o local onde o projeto foi desenvolvido.
	
As Aplicações Web consistem num programa python que serve conteúdos estáticos, 
esta parte do trabalho serve como ponte para todos os outros componentes, para além disto tem também objetivos próprios a ser 
explicados mais à frente na \autoref{chap.aplicacao}.

A parte da Persistência consiste numa base de dados relacional e obtenção de dados da mesma, 
tudo isto será depois utilizado pela parte das Aplicações Web para registar as músicas criadas, os excertos que existem e os 
votos de cada um dos utilizadores, explicado em maior detalhe em \autoref{chap.persistencia}.

A parte do Gerador de Música deverá aceitar uma pauta na forma de um dicionário, gerando a música
especificada. Caso a geração tenha sucesso, a música deverá ser escrita no sistema de
ficheiros. Deverá igualmente ser gerado um identificador com base no conteúdo, sendo
este devolvido pelo módulo para persistência na base de dados.
Este componente deverá suportar pelo menos 5 efeitos a aplicar no momento da geração
da música.

\chapter{Desenvolvimento}
\label{chap.desenvolvimento}
	
\section{Interface Web}
\label{chap.interface}
..............

\section{Aplicação Web}
\label{chap.aplicacao}
A aplicação Web é o programa em python que serve como ponte entre todas as partes do projeto.
Tem as funções list, get, sheet e vote:

\subsection{list}
A função \textit{list} está dividida nas \textit{samples} e nas \textit{songs}.
\paragraph{Samples}
Este módulo quando usado vai devolver o JSON de todas as samples com o 
\textit{name, date, id, length} das samples guardado na base de dados na tabela de Samples.
\paragraph{Songs}
À semelhança do módulo das Samples, este irá devolver o JSON de todas 
as músicas criadas com o \textit{name, date, id, length, uses, votes, author}
 que também se encontra guardado na base de dados.
 \paragraph{Aceder}
	Para aceder deve-se escrever \textit{/list?type=songs} para o JSON de músicas e \textit{/list?type=samples}
 para os das samples
 \subsection{get}
 Esta função devolve todos os atributos de uma sample ou song através do seu id.
 \p Não se encontra funcional.
 \paragraph{Aceder} 
 Para aceder deve-se escrever \textit{/get?id=id} em que o id é uma string de 16 caracteres de 0 a 9 e de A a F.

 \subsection{sheet}
 Esta função deveria fazer a mistura das samples para criar músicas.
 \p Não se encontra funcional.

 \subsection{vote}
Função onde recebendo o id da música e os pontos(-1 ou 1) adiciona 
um voto negativo ou positivo na música.
A função averigua se o utilizador já votou ou não. Caso já tenha votado
dá para tirar o voto clicando no botão contrário ao usado para votar e
poderá votar denovo.
A função adiciona os votos diretamente na base de dados.

................

\section{Persistência}
\label{chap.persistencia}
...............

\section{Gerador de Músicas}
\label{chap.gerador}
...................

	 
\chapter{Conclusões Finais}
\label{chap.conclusao}
	Com este projeto de gerador de músicas foi possível consolidar os conhecimentos 
	adquiridos em aulas pois foi necessário tudo o que foi realizado ao longo semestre. 
	
	Infelizmente, o projeto não está completamente funcional. Ainda assim, com o que foi feito, mudou a forma como os 
	elementos do grupo veem as aplicações como esta de gerador de músicas, todo o trabalho que está por detrás do que é 
	apresentado ao utilizador.  

 
\chapter{Contribuições dos autores}
\label{contribuições}

\noindent
André Alvez - 0\% \\
Daniel Correia - 0\% \\
Pedro Almeida - 0\% \\
Pedro Valente - 0\% 


\end{document}